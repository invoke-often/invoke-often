\addchap{Liber Iod}\index{Grade Studies \& Work!Dominus Liminis}
  \chapnum{DCCCXXXI\footnote{\textgreek{φαλλος}. From the Equinox Vol. 1, No. 7; Magick in Theory and Practice}}


(This book was formerly called \textsc{Vesta}. It is referred to the path of Virgo and the letter Yod.)\editorsnote{This was labelled Liber TAV, Class B, in the text of Equinox Vol. 1 No. 7. Named Liber Yod in Magick in Theory and Practice.}

\addsec*{I.}

1. This is the Book of drawing all to a point.

2. Herein are described three methods whereby the consciousness of the Many may be melted to that of the One.

\addsec*{II}

\addsec*{First Method}

0. Let a magical circle be constructed, and within it an upright Tau drawn upon the ground. Let this Tau be divided into 10 squares (see Liber CMLXIII.\editorsnote{Liber \textgreek{Θεσαυρου Ειδωλων}, \enquote{The Treasure House of Images}, in the Equinox Vol. I, No. III}, Illustration 1).

1. Let the Magician be armed with the Sword of Art.\footnotemark

2. Let him wear the black robe of a Neophyte.

3. Let a single small flame of camphor burn at the top of the Tau, and let there be no other light or ornament.\footnotemark[\value{footnote}]\footnotetext{This ritual is preferably performed by the Adept as an Hermit armed with wand and lamp, instead of as in text.\textemdash{}N. \textsc{[Equinox]}}

4. Let him \enquote{open} the Temple as in DCLXXI., or in any other convenient manner.

5. Standing at the appropriate quarters, at the edge of the circle, let him banish the 5 elements by the appropriate rituals.

6. Standing at the edge of the circle, let him banish the 7 planets by the appropriate rituals. Let him face the actual position of each planet in the heavens at the time of his working.

7. Let him further banish the twelve signs of the Zodiac by the appropriate rituals, facing each sign in turn.

8. Let him at each of these 24 banishings make three circuits widdershins, with the signs of Horus and Harpocrates in the East as he passes it.

9. Let him advance to the square of Malkuth in the Tau, and perform a ritual of banishing Malkuth. But here let him not leave not the square to circumambulate the circle, but use the formula and God-form of Harpocrates.

10. Let him advance in turn to the squares Yesod, Hod, Netzach, Tiphereth, Geburah, Chesed, and banish each by appropriate rituals.

11. And let him know that such rituals include the pronunciation of the appropriate names of God backwards, and also a curse against the Sephira in respect of all that which it is, for that it is that which distinguishes and separates it from Kether.

12. Advancing to the squares of Binah and Chokmah in turn, let him banish these also. And for that by now an awe and trembling shall have taken hold upon him, let him banish these by a supreme ritual of inestimable puissance. And let him beware exceedingly lest his will falter, or his courage fail.

13. Finally, let him, advancing to the square of Kether, banish that also by what means he may. At the end whereof let him set his foot upon the light, extinguishing it;\footnote{If armed with wand and lamp, let him extinguish the light with his hand. \textemdash{} N. \textsc{[Equinox]}} and, as he falleth, let him fall within the circle.

\addsec*{Second Method}

1. Let the Hermit be seated in his Asana, robed, and let him meditate in turn upon ever several part of his body until that part is so unreal to him that he no longer includes it in his comprehension of himself. For example, if it be his right foot, let him touch that foot, and be alarmed, thinking, \enquote{A foot! What is this foot? Surely I am alone in the Hermitage!}

And this practice should be carried out not only at the time of meditation, but during the day's work.

2. This meditation is to be assisted by reasoning; as, \enquote{This foot is not I. If I should lose my foot, I should still be I. This foot is a mass of changing and decaying flesh, bone, skin, blood, lymph, etc., while I am the Unchanging and Immortal Spirit, uniform, not made, unbegotten, formless, self-luminous}, etc.

3. This practice being perfect for each part of the body, let him combine his workings until the whole body is thus understood as the non-Ego and as illusion.

4. Let then the Hermit, seated in his Asana, meditate upon the Muladhara chakra and its correspondence as a power of the mind, and destroy it in the same manner as aforesaid. Also by reasoning: \enquote{This emotion (memory, imagination, intellect, will, as it may be) is not I. This emotion is transient: I am immovable. This emotion is passion; I am peace} And so on.

Let the other Chakras in their turn be thus destroyed, each one with its mental or moral attribute.

5. In this let him be aided by his own psychological analysis, so that no part of his conscious being be thus left undestroyed. And on his thoroughness in this matter may turn his success.

6. Lastly, having drawn all his being into the highest Sahasrara Chakra, let him remain eternally fixed in meditation thereupon.

7. Aum.

\addsec*{Third Method}

1. Let the Hermit stimulate each of the senses in turn, concentrating upon each until it ceases to stimulate.

[The senses of sight and touch are extremely difficult to conquer. In the end the Hermit must be utterly unable by any effort to see or feel the object of those senses. O.M. \textsc{[Equinox]}]

2.  This being perfected, let him combine them two at a time.

For example, let him chew ginger (taste and touch), and watch a waterfall (sight and hearing), and watch incense (sight and smell), and crunch sugar in his teeth (taste and hearing), and so on.

3. These twenty-five practices being accomplished, let him combined three at a time, then four at a time.

4. Lastly, let him combine all the senses in a single object.

And herein may a sixth sense be included. He is then to withdraw himself entirely from all these stimulations, \textit{perinde ac cadaver},\editorsnote{\enquote{in the manner of a corpse}} in spite of his own efforts to attach himself to them.

5. By this method it is said that the demons of the Ruach, that is, thoughts and memories, are inhibited, and We deny it not. But if so be that they arise, let him build a wall between himself and them according to the method.

6. Thus having stilled the voices of the Six, may he sense the subtlety of the Seventh.

7. Aum.\editorsnote{\Aumgn{} in Magick in Theory and Practice.}

\addsec*{[Fourth Method]}

\textit{[We add the following, contributed by a friend at that time without the \Argentium{} and its dependent orders. He worked out the method himself, and we think it may prove useful to many. O.M. \textsc{[Equinox]}]}

(1) The beginner must first practice breathing regularly through the nose, at the same time trying hard to imagine that the breath goes to the Ajna and not to the lungs.

The pranayama exercises described in Equinox, Vol. I., No. 4, p. 101\editorsnote{\enquote{The Temple of Solomon the King IV; Yogas}.}, must next be practiced, always with the idea that Ajna is breathing.

Try to realise that \textit{power}, not air, is being drawn into the Ajna, is being concentrated there during Kumbhaka\index{Kumbhaka}, and is vivifying the Ajna during expiration. Try rather to increase the force of concentration in the Ajna than to increase excessively the length of Kumbhaka, as this is dangerous if rashly undertaken.

(2) Walk slowly in a quiet place; realise that the legs are moving, and study their movements. Understand thoroughly that these movements are due to nerve messages sent down from the brain, and that the controlling power lies in the Ajna. The legs are automatic, like those of a wooden monkey: the power in the Ajna is that which does the work, is that which walks.

Apply this method to every other muscular movement.

(3) Lie flat on the back with the feet under a heavy piece of furniture. Keeping the spine straight and the arms in line with the body, rise slowly to a sitting posture, by means of the force residing in the Ajna (\textit{i.e. try to prevent the mind dwelling on any other exertion or sensation}).

Then let the body slowly down to its original position. Repeat this two or three times every night and morning, and slowly increase the number of repetitions.

(4) Try to transfer all bodily sensations to the Ajna: \textit{e.g.,} \enquote{I am cold} should mean \enquote{I feel cold} or, better still, \enquote{I am aware of the sensation of cold} \textemdash{} transfer this to the Ajna, \enquote{the Ajna is aware}, etc.

(5) Pain if very slight may easily be transferred to the Ajna after a little practice. The best method for a beginner is to \textit{imagine} he has a pain in the body and then imagine that it passes directly into the Ajna. It does not pass through the intervening structures, but goes direct. After continual practice even severe pain may be transferred to the Ajna.

(6) Fix the mind on the base of the spine and then gradually move the thoughts upwards to the Ajna.

(In this meditation Ajna is the Holy of Holies, but it is dark and empty.)

Finally strive hard to drive anger and other obsessing thoughts into the Ajna. Try to develop a tendency to think hard of Ajna when these thoughts attack the mind, and let Ajna conquer them.

Beware of thinking \enquote{\textit{my} Ajna.} In these meditations and practices, Ajna does not belong to you; Ajna is the master and the worker, you are the wooden monkey.
